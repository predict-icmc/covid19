% Options for packages loaded elsewhere
\PassOptionsToPackage{unicode}{hyperref}
\PassOptionsToPackage{hyphens}{url}
%
\documentclass[
]{article}
\usepackage{lmodern}
\usepackage{amssymb,amsmath}
\usepackage{ifxetex,ifluatex}
\ifnum 0\ifxetex 1\fi\ifluatex 1\fi=0 % if pdftex
  \usepackage[T1]{fontenc}
  \usepackage[utf8]{inputenc}
  \usepackage{textcomp} % provide euro and other symbols
\else % if luatex or xetex
  \usepackage{unicode-math}
  \defaultfontfeatures{Scale=MatchLowercase}
  \defaultfontfeatures[\rmfamily]{Ligatures=TeX,Scale=1}
\fi
% Use upquote if available, for straight quotes in verbatim environments
\IfFileExists{upquote.sty}{\usepackage{upquote}}{}
\IfFileExists{microtype.sty}{% use microtype if available
  \usepackage[]{microtype}
  \UseMicrotypeSet[protrusion]{basicmath} % disable protrusion for tt fonts
}{}
\makeatletter
\@ifundefined{KOMAClassName}{% if non-KOMA class
  \IfFileExists{parskip.sty}{%
    \usepackage{parskip}
  }{% else
    \setlength{\parindent}{0pt}
    \setlength{\parskip}{6pt plus 2pt minus 1pt}}
}{% if KOMA class
  \KOMAoptions{parskip=half}}
\makeatother
\usepackage{xcolor}
\IfFileExists{xurl.sty}{\usepackage{xurl}}{} % add URL line breaks if available
\IfFileExists{bookmark.sty}{\usepackage{bookmark}}{\usepackage{hyperref}}
\hypersetup{
  pdftitle={Compara-modelos},
  pdfauthor={predict-icmc},
  hidelinks,
  pdfcreator={LaTeX via pandoc}}
\urlstyle{same} % disable monospaced font for URLs
\usepackage[margin=1in]{geometry}
\usepackage{color}
\usepackage{fancyvrb}
\newcommand{\VerbBar}{|}
\newcommand{\VERB}{\Verb[commandchars=\\\{\}]}
\DefineVerbatimEnvironment{Highlighting}{Verbatim}{commandchars=\\\{\}}
% Add ',fontsize=\small' for more characters per line
\usepackage{framed}
\definecolor{shadecolor}{RGB}{248,248,248}
\newenvironment{Shaded}{\begin{snugshade}}{\end{snugshade}}
\newcommand{\AlertTok}[1]{\textcolor[rgb]{0.94,0.16,0.16}{#1}}
\newcommand{\AnnotationTok}[1]{\textcolor[rgb]{0.56,0.35,0.01}{\textbf{\textit{#1}}}}
\newcommand{\AttributeTok}[1]{\textcolor[rgb]{0.77,0.63,0.00}{#1}}
\newcommand{\BaseNTok}[1]{\textcolor[rgb]{0.00,0.00,0.81}{#1}}
\newcommand{\BuiltInTok}[1]{#1}
\newcommand{\CharTok}[1]{\textcolor[rgb]{0.31,0.60,0.02}{#1}}
\newcommand{\CommentTok}[1]{\textcolor[rgb]{0.56,0.35,0.01}{\textit{#1}}}
\newcommand{\CommentVarTok}[1]{\textcolor[rgb]{0.56,0.35,0.01}{\textbf{\textit{#1}}}}
\newcommand{\ConstantTok}[1]{\textcolor[rgb]{0.00,0.00,0.00}{#1}}
\newcommand{\ControlFlowTok}[1]{\textcolor[rgb]{0.13,0.29,0.53}{\textbf{#1}}}
\newcommand{\DataTypeTok}[1]{\textcolor[rgb]{0.13,0.29,0.53}{#1}}
\newcommand{\DecValTok}[1]{\textcolor[rgb]{0.00,0.00,0.81}{#1}}
\newcommand{\DocumentationTok}[1]{\textcolor[rgb]{0.56,0.35,0.01}{\textbf{\textit{#1}}}}
\newcommand{\ErrorTok}[1]{\textcolor[rgb]{0.64,0.00,0.00}{\textbf{#1}}}
\newcommand{\ExtensionTok}[1]{#1}
\newcommand{\FloatTok}[1]{\textcolor[rgb]{0.00,0.00,0.81}{#1}}
\newcommand{\FunctionTok}[1]{\textcolor[rgb]{0.00,0.00,0.00}{#1}}
\newcommand{\ImportTok}[1]{#1}
\newcommand{\InformationTok}[1]{\textcolor[rgb]{0.56,0.35,0.01}{\textbf{\textit{#1}}}}
\newcommand{\KeywordTok}[1]{\textcolor[rgb]{0.13,0.29,0.53}{\textbf{#1}}}
\newcommand{\NormalTok}[1]{#1}
\newcommand{\OperatorTok}[1]{\textcolor[rgb]{0.81,0.36,0.00}{\textbf{#1}}}
\newcommand{\OtherTok}[1]{\textcolor[rgb]{0.56,0.35,0.01}{#1}}
\newcommand{\PreprocessorTok}[1]{\textcolor[rgb]{0.56,0.35,0.01}{\textit{#1}}}
\newcommand{\RegionMarkerTok}[1]{#1}
\newcommand{\SpecialCharTok}[1]{\textcolor[rgb]{0.00,0.00,0.00}{#1}}
\newcommand{\SpecialStringTok}[1]{\textcolor[rgb]{0.31,0.60,0.02}{#1}}
\newcommand{\StringTok}[1]{\textcolor[rgb]{0.31,0.60,0.02}{#1}}
\newcommand{\VariableTok}[1]{\textcolor[rgb]{0.00,0.00,0.00}{#1}}
\newcommand{\VerbatimStringTok}[1]{\textcolor[rgb]{0.31,0.60,0.02}{#1}}
\newcommand{\WarningTok}[1]{\textcolor[rgb]{0.56,0.35,0.01}{\textbf{\textit{#1}}}}
\usepackage{graphicx,grffile}
\makeatletter
\def\maxwidth{\ifdim\Gin@nat@width>\linewidth\linewidth\else\Gin@nat@width\fi}
\def\maxheight{\ifdim\Gin@nat@height>\textheight\textheight\else\Gin@nat@height\fi}
\makeatother
% Scale images if necessary, so that they will not overflow the page
% margins by default, and it is still possible to overwrite the defaults
% using explicit options in \includegraphics[width, height, ...]{}
\setkeys{Gin}{width=\maxwidth,height=\maxheight,keepaspectratio}
% Set default figure placement to htbp
\makeatletter
\def\fps@figure{htbp}
\makeatother
\setlength{\emergencystretch}{3em} % prevent overfull lines
\providecommand{\tightlist}{%
  \setlength{\itemsep}{0pt}\setlength{\parskip}{0pt}}
\setcounter{secnumdepth}{-\maxdimen} % remove section numbering

\title{Compara-modelos}
\author{predict-icmc}
\date{14/12/2020}

\begin{document}
\maketitle

\hypertarget{lendo-os-dados}{%
\subsection{Lendo os dados}\label{lendo-os-dados}}

Para obter os dados mais atuais do Covid-19 do site \texttt{brasil.io}
utilizamos o pacote \texttt{Gather-Covid}

\begin{Shaded}
\begin{Highlighting}[]
\CommentTok{# Para instalar utilize o comando}
\CommentTok{#remotes::install_github("predict-icmc/gather-data")}

\KeywordTok{library}\NormalTok{(tidyverse)}
\end{Highlighting}
\end{Shaded}

\begin{verbatim}
## -- Attaching packages --------------------------------------- tidyverse 1.3.0 --
\end{verbatim}

\begin{verbatim}
## v ggplot2 3.3.3     v purrr   0.3.4
## v tibble  3.0.4     v dplyr   1.0.2
## v tidyr   1.1.2     v stringr 1.4.0
## v readr   1.4.0     v forcats 0.5.0
\end{verbatim}

\begin{verbatim}
## -- Conflicts ------------------------------------------ tidyverse_conflicts() --
## x dplyr::filter() masks stats::filter()
## x dplyr::lag()    masks stats::lag()
\end{verbatim}

\begin{Shaded}
\begin{Highlighting}[]
\KeywordTok{library}\NormalTok{(gather.covid)}

\NormalTok{df <-}\StringTok{ }\KeywordTok{pegaCorona}\NormalTok{(}\DataTypeTok{tipo =} \StringTok{"caso_full"}\NormalTok{)}
\end{Highlighting}
\end{Shaded}

\begin{verbatim}
## [1] "Fazendo o download...."
## [1] "Download concluido. Transformando os dados"
\end{verbatim}

\begin{Shaded}
\begin{Highlighting}[]
\NormalTok{selectedCity <-}\StringTok{ }\NormalTok{df }\OperatorTok\StringTok{ }\KeywordTok{filter}\NormalTok{(city }\OperatorTok{==}\StringTok{ "São Paulo"}\NormalTok{)}
\end{Highlighting}
\end{Shaded}

\begin{Shaded}
\begin{Highlighting}[]
\CommentTok{#---- testes Fla}
\KeywordTok{library}\NormalTok{(forecast)}
\end{Highlighting}
\end{Shaded}

\begin{verbatim}
## Registered S3 method overwritten by 'quantmod':
##   method            from
##   as.zoo.data.frame zoo
\end{verbatim}

\begin{Shaded}
\begin{Highlighting}[]
\CommentTok{#--- Conjuntos Treino e Teste}
\NormalTok{treino<-selectedCity}\OperatorTok{$}\NormalTok{new_confirmed[}\DecValTok{1}\OperatorTok{:}\DecValTok{276}\NormalTok{]}
\NormalTok{test<-selectedCity}\OperatorTok{$}\NormalTok{new_confirmed[}\DecValTok{277}\OperatorTok{:}\DecValTok{290}\NormalTok{]}

\NormalTok{modelo <-}\StringTok{ }\KeywordTok{auto.arima}\NormalTok{(treino,}
                    \DataTypeTok{trace =}\NormalTok{ T, }\CommentTok{# habilitando o display para acompanhar}
                    \DataTypeTok{stepwise =}\NormalTok{ F, }\CommentTok{# permitindo uma busca mais profunda}
                    \DataTypeTok{approximation =}\NormalTok{ F)}
\end{Highlighting}
\end{Shaded}

\begin{verbatim}
## 
##  ARIMA(0,1,0)                    : 4731.681
##  ARIMA(0,1,0) with drift         : 4733.707
##  ARIMA(0,1,1)                    : 4618.448
##  ARIMA(0,1,1) with drift         : 4619.882
##  ARIMA(0,1,2)                    : 4611.802
##  ARIMA(0,1,2) with drift         : 4613.271
##  ARIMA(0,1,3)                    : 4613.606
##  ARIMA(0,1,3) with drift         : 4615.082
##  ARIMA(0,1,4)                    : 4609.186
##  ARIMA(0,1,4) with drift         : 4610.645
##  ARIMA(0,1,5)                    : 4598.89
##  ARIMA(0,1,5) with drift         : 4600.411
##  ARIMA(1,1,0)                    : 4696.302
##  ARIMA(1,1,0) with drift         : 4698.335
##  ARIMA(1,1,1)                    : 4612.822
##  ARIMA(1,1,1) with drift         : 4614.294
##  ARIMA(1,1,2)                    : 4613.766
##  ARIMA(1,1,2) with drift         : 4615.247
##  ARIMA(1,1,3)                    : Inf
##  ARIMA(1,1,3) with drift         : Inf
##  ARIMA(1,1,4)                    : 4598.219
##  ARIMA(1,1,4) with drift         : 4599.89
##  ARIMA(2,1,0)                    : 4682.025
##  ARIMA(2,1,0) with drift         : 4684.066
##  ARIMA(2,1,1)                    : 4612.394
##  ARIMA(2,1,1) with drift         : 4613.864
##  ARIMA(2,1,2)                    : 4594.84
##  ARIMA(2,1,2) with drift         : 4596.459
##  ARIMA(2,1,3)                    : 4596.13
##  ARIMA(2,1,3) with drift         : 4597.729
##  ARIMA(3,1,0)                    : 4675.687
##  ARIMA(3,1,0) with drift         : 4677.738
##  ARIMA(3,1,1)                    : 4608.242
##  ARIMA(3,1,1) with drift         : 4609.702
##  ARIMA(3,1,2)                    : 4595.862
##  ARIMA(3,1,2) with drift         : 4597.453
##  ARIMA(4,1,0)                    : 4655.934
##  ARIMA(4,1,0) with drift         : 4657.99
##  ARIMA(4,1,1)                    : 4594.808
##  ARIMA(4,1,1) with drift         : 4596.264
##  ARIMA(5,1,0)                    : 4623.99
##  ARIMA(5,1,0) with drift         : 4626.004
## 
## 
## 
##  Best model: ARIMA(4,1,1)
\end{verbatim}

\begin{Shaded}
\begin{Highlighting}[]
\KeywordTok{print}\NormalTok{(modelo) }\CommentTok{# exibindo os parametros do modelo}
\end{Highlighting}
\end{Shaded}

\begin{verbatim}
## Series: treino 
## ARIMA(4,1,1) 
## 
## Coefficients:
##          ar1      ar2      ar3      ar4      ma1
##       0.0676  -0.1325  -0.1520  -0.2506  -0.8416
## s.e.  0.0648   0.0603   0.0597   0.0617   0.0353
## 
## sigma^2 estimated as 1021197:  log likelihood=-2291.25
## AIC=4594.49   AICc=4594.81   BIC=4616.2
\end{verbatim}

\begin{Shaded}
\begin{Highlighting}[]
\CommentTok{## avaliando os resíduos}
\KeywordTok{checkresiduals}\NormalTok{(modelo) }
\end{Highlighting}
\end{Shaded}

\includegraphics{compara-modelos_files/figure-latex/unnamed-chunk-2-1.pdf}

\begin{verbatim}
## 
##  Ljung-Box test
## 
## data:  Residuals from ARIMA(4,1,1)
## Q* = 14.389, df = 5, p-value = 0.01332
## 
## Model df: 5.   Total lags used: 10
\end{verbatim}

\begin{Shaded}
\begin{Highlighting}[]
\KeywordTok{shapiro.test}\NormalTok{(modelo}\OperatorTok{$}\NormalTok{residuals)}
\end{Highlighting}
\end{Shaded}

\begin{verbatim}
## 
##  Shapiro-Wilk normality test
## 
## data:  modelo$residuals
## W = 0.89759, p-value = 9.983e-13
\end{verbatim}

\begin{Shaded}
\begin{Highlighting}[]
\KeywordTok{var}\NormalTok{(modelo}\OperatorTok{$}\NormalTok{residuals) }\CommentTok{# variancia alta  }
\end{Highlighting}
\end{Shaded}

\begin{verbatim}
## [1] 1000408
\end{verbatim}

\begin{Shaded}
\begin{Highlighting}[]
\KeywordTok{mean}\NormalTok{(modelo}\OperatorTok{$}\NormalTok{residuals)}\CommentTok{# }
\end{Highlighting}
\end{Shaded}

\begin{verbatim}
## [1] 47.04737
\end{verbatim}

\begin{Shaded}
\begin{Highlighting}[]
\CommentTok{# com isso pode-se concluir que não foi criado um bom modelo de previsão }

\NormalTok{previsao <-}\StringTok{ }\KeywordTok{forecast}\NormalTok{(modelo, }\DataTypeTok{h =} \DecValTok{14}\NormalTok{) }\CommentTok{#duas semanas}
\KeywordTok{print}\NormalTok{(previsao)}
\end{Highlighting}
\end{Shaded}

\begin{verbatim}
##     Point Forecast        Lo 80    Hi 80     Lo 95    Hi 95
## 277       1490.840  195.7773968 2785.903 -489.7874 3471.468
## 278       1228.733  -98.9967493 2556.464 -801.8547 3259.322
## 279       1102.609 -226.1923667 2431.410 -929.6172 3134.835
## 280       1440.230  111.1581705 2769.302 -592.4099 3472.870
## 281       1449.218  108.9714951 2789.464 -600.5121 3498.948
## 282       1489.931  144.7073090 2835.154 -567.4110 3547.273
## 283       1471.789  107.6137297 2835.963 -614.5368 3558.114
## 284       1379.196   -5.6101055 2764.001 -738.6819 3497.073
## 285       1366.903  -34.7960814 2768.601 -776.8106 3510.616
## 286       1370.898  -36.3532371 2778.149 -781.3069 3523.102
## 287       1391.415  -18.8338859 2801.663 -765.3743 3548.203
## 288       1417.343    3.3963124 2831.290 -745.1019 3579.788
## 289       1418.850   -0.7505563 2838.450 -752.2415 3589.941
## 290       1411.396  -16.6340360 2839.426 -772.5876 3595.380
\end{verbatim}

\begin{Shaded}
\begin{Highlighting}[]
\KeywordTok{autoplot}\NormalTok{(previsao)}
\end{Highlighting}
\end{Shaded}

\includegraphics{compara-modelos_files/figure-latex/unnamed-chunk-2-2.pdf}

\begin{Shaded}
\begin{Highlighting}[]
\CommentTok{#- comparar a previsao com o conj teste}
\end{Highlighting}
\end{Shaded}

\end{document}
